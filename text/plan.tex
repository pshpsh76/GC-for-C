\section{План дальнейшей работы}

\subsection{Этап 1: Разработка базового GC (Mark-Sweep)}
\begin{itemize}
    \item Проектирование архитектуры библиотеки GC.
    \item Определение API библиотеки:
    \begin{itemize}
        \item \texttt{gc\_init} – инициализация GC и передача GC roots (глобальные переменные, стеки).
        \item \texttt{gc\_malloc} – выделение памяти под объекты.
        \item \texttt{gc\_collect} – сборка мусора с использованием алгоритма Mark-Sweep.
    \end{itemize}
    \item Реализация механизма отслеживания GC roots:
    \begin{itemize}
        \item Определение структуры данных для хранения GC roots.
        \item Реализация механизма регистрации и удаления GC roots.
    \end{itemize}
    \item Реализация алгоритма \textbf{Mark-Sweep}:
    \begin{itemize}
        \item Фаза \textbf{Mark}: обход объектов от GC roots и пометка достижимых.
        \item Фаза \textbf{Sweep}: освобождение неотмеченных объектов.
    \end{itemize}
    \item Разработка базовых тестов для проверки работы GC.
    \item Оптимизация и исправление ошибок.
\end{itemize}

\subsection{Этап 2: Добавление триггеров для автоматического запуска GC}
\begin{itemize}
    \item Определение критериев для автоматического запуска \texttt{gc\_collect}:
    \begin{itemize}
        \item Заполнение выделенной памяти (например, если выделено больше X байт).
        \item Временные интервалы (например, сборка каждые N мс).
        \item Количество аллокаций (\texttt{gc\_malloc} вызывается определённое число раз).
    \end{itemize}
    \item Реализация триггеров.
    \item Оптимизация частоты вызова GC (чтобы не снижалась производительность программы).
    \item Написание тестов для проверки работы триггеров.
\end{itemize}

\subsection{Этап 3: Поддержка многопоточности}
\begin{itemize}
    \item Анализ возможных проблем многопоточного доступа
    \item Использование мьютексов или написание lock free версии
    \item Реализация потокобезопасной версии \texttt{gc\_malloc} и \texttt{gc\_collect}.
    \item Тестирование многопоточной работы GC.
\end{itemize}

\subsection{Этап 4: Дальнейшее развитие GC}
\textbf{Вариант 1:} Встраивание GC в собственный интерпретатор или компилятор.
\begin{itemize}
    \item Разработка простого интерпретатора (или интеграция с существующим).
    \item Адаптация GC под работу с интерпретатором.
\end{itemize}

\textbf{Вариант 2:} Реализация более сложного алгоритма GC:
\begin{itemize}
    \item \textbf{Mark-Compact}: реализация алгоритма с дефрагментацией памяти.
    \item \textbf{Generational GC}: реализация поколенческого GC с разделением объектов на молодое и старое поколения.
\end{itemize}

\textbf{Вариант 3:} Оптимизация производительности:
\begin{itemize}
    \item Улучшение стратегии выделения памяти.
    \item Оптимизация стадий Mark, Sweep
    \item Тестирование других алгоритмов
\end{itemize}
