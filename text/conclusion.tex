\section{Заключение}

В рамках курсового проекта был реализован полностью функциональный сборщик мусора, предназначенный для использования в проектах на языке C/C++. Работа охватывала как теоретическое исследование существующих алгоритмов, так и практическую реализацию, профилирование, оптимизацию и экспериментальную валидацию решения.

\subsection*{Основные достижения проекта}

\begin{itemize}
    \item Разработана масштабируемая архитектура сборщика мусора, включающая независимые компоненты:
    \begin{itemize}
        \item \texttt{GCImpl} — управление аллокациями и памятью;
        \item \texttt{GCScheduler} — контроль авто-сборки;
        \item \texttt{GCPacer} — логика авто-сборки.
    \end{itemize}
    \item Реализован эффективный Mark-Sweep GC с полной поддержкой многопоточности и ручным управлением корнями.
    \item Введены \textbf{safepoint}-механизмы и stop-the-world-сборка с корректной синхронизацией потоков.
    \item Проведено масштабное профилирование и оптимизация.
    \item Поддержан гибкий runtime-интерфейс управления GC (ручной и автоматический режим, установка порогов, интервалы и прочее).
    \item Разработан полноценный стек тестов: юнит-тесты, многопоточные сценарии, нагрузочные бенчмарки.
\end{itemize}

\subsection*{Практическая значимость}

Проект представляет интерес как для встраивания в прикладные C++ проекты, так и в качестве базы для разработки языков программирования, интерпретаторов и образовательных инструментов. Он демонстрирует, что эффективный сборщик мусора может быть реализован без поддержки со стороны компилятора и без необходимости использования внешней виртуальной машины.

\subsection*{Ограничения и перспективы развития}

Несмотря на достигнутые результаты, работа GC ограничена следующими моментами:

\begin{itemize}
    \item используется только один алгоритм (Mark-Sweep), без поддержки компактизации или поколенческого подхода;
    \item сборка осуществляется в stop-the-world-режиме;
    \item управление корнями требует ручной и явной регистрации.
\end{itemize}

В качестве перспектив возможны следующие направления развития:

\begin{itemize}
    \item внедрение поколенческого GC;
    \item добавление поддержки копирующих или компактифицирующих алгоритмов;
    \item экспериментальная реализация инкрементальной или concurrent-сборки;
    \item интеграция с компилятором/анализатором для автоматического сбора корней;
    \item расширение API для поддержки слабых ссылок, финализаторов, внешней трассировки.
\end{itemize}

\subsection*{Итог}

Разработанный сборщик мусора демонстрирует высокую производительность, масштабируемость, корректность и простоту интеграции. Проект может служить как практическим инструментом, так и основой для дальнейших исследований в области автоматического управления памятью.
