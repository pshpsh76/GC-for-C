\section{Сравнение с аналогами и обоснование новизны}

\subsection{Существующие решения}

Механизмы сборки мусора присутствуют во многих современных языках программирования и виртуальных машинах. Ниже перечислены основные существующие реализации:

\begin{itemize}
    \item \textbf{CPython (Python)} использует подсчёт ссылок (\texttt{Reference Counting}) как основной механизм, дополнительно реализуя поколенческий сборщик мусора для циклических ссылок.
    \item \textbf{Java Virtual Machine (JVM)} поддерживает множество GC-алгоритмов: Serial GC, G1, Shenandoah, ZGC, включая concurrent и generational GC. Реализация тесно связана с байткодом и компилятором.
    \item \textbf{.NET (C\#)} использует поколенческий Mark-Compact GC с конкурентными возможностями для серверных приложений.
    \item \textbf{Go} применяет concurrent Mark-Sweep GC, заточенный под low-latency сценарии.
    \item \textbf{Rust / C++} предоставляют только ручное управление памятью, иногда с использованием умных указателей (\texttt{shared\_ptr}, \texttt{Rc}, \texttt{Arc}), что не является полноценным GC.
\end{itemize}

Во всех перечисленных случаях реализация GC встроена в компилятор или рантайм, тесно завязана на типовую информацию, байткод и внутреннюю структуру языка. Это делает их либо недоступными, либо избыточными для внедрения в проекты на C/C++.

\subsection{Позиционирование данного проекта}

Разработанный сборщик мусора представляет собой самостоятельную, независимую библиотеку GC с ручной интеграцией в проекты на C и C++. Его ключевые отличия:

\begin{itemize}
    \item \textbf{Не требует поддержки со стороны компилятора}. Полностью работает в пространстве пользовательского кода, не нуждается в знании типов и метаинформации.
    \item \textbf{Реализует stop-the-world Mark-Sweep GC} с многопоточной параллельной маркировкой и ручным управлением корнями.
    \item \textbf{Имеет гибкий runtime-интерфейс} для настройки порогов авто-GC и контроля сборки в процессе выполнения программы.
    \item \textbf{Полноценная многопоточность с учётом safepoint'ов и синхронизации}.
    \item \textbf{Полный тестовый и бенчмаркинговый стек}: от юнитов до стрессов, от функциональных проверок до метрик throughput.
\end{itemize}

\subsection{Новизна и вклад}

Несмотря на то, что базовый алгоритм Mark-Sweep хорошо известен, представленный проект обладает рядом аспектов, отличающих его от стандартных реализаций:

\begin{enumerate}[label=\arabic*.]
    \item \textbf{Форм-фактор:} легко встраиваемая C/C++-библиотека, ориентированная на независимое использование без JVM/CLR.
    \item \textbf{Архитектура:} модульное разбиение на GCImpl / GCScheduler / GCPacer облегчает расширение и модификации.
    \item \textbf{Высокоэффективная реализация:}
    \begin{itemize}
        \item оптимизированный поиск \texttt{FindAllocation} с ускорением;
        \item эвристики фильтрации и кэширования;
        \item частичная сортировка аллокаций с линейным слиянием.
    \end{itemize}
    \item \textbf{Доказанная эффективность:} бенчмарки демонстрируют сборку 10\,000 объектов менее чем за 40 мкс, пропускную способность до 23 млн объектов/сек.
    \item \textbf{Исследовательский вклад:} проект показывает, как можно реализовать эффективный GC без поддержки типов и без связки с JIT/VM, используя лишь стандартные механизмы языка.
\end{enumerate}

\subsection{Возможности повторного использования}

Разработанная библиотека может быть применена в:

\begin{itemize}
    \item Создании интерпретаторов или DSL-языков,\\ где требуется простое автоматическое управление памятью.
    \item Встраиваемых системах или проектах на C++, нуждающихся в безопасной и автоматической очистке памяти без утечек.
    \item Образовательных целях — как демонстрация реализации базового и расширяемого GC.
    \item Исследовательских работах, связанных с сравнением GC-стратегий, \\оптимизацией runtime-алгоритмов и синхронизацией потоков.
\end{itemize}
