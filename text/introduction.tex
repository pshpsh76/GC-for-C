\section{Введение}


Автоматическое управление памятью играет ключевую роль в современных языках программирования, позволяя разработчикам создавать сложные программные системы без необходимости вручную отслеживать выделение и освобождение памяти. Один из наиболее важных компонентов этого процесса — \textbf{сборщик мусора} (garbage collector, GC), который автоматически удаляет неиспользуемые объекты, предотвращая утечки памяти и обеспечивая эффективное использование ресурсов. 

Сборка мусора особенно актуальна в высокоуровневых языках программирования, таких как Python, Java, C\#, Go, где управление памятью скрыто от разработчика. Однако низкоуровневые языки, такие как C и C++, традиционно требуют ручного управления памятью, что увеличивает вероятность ошибок, таких как утечки памяти и использование уже освобождённых объектов.

\subsection{Постановка задачи}

Цель данного проекта — \textbf{изучение актуальных алгоритмов сборки мусора} и \textbf{разработка собственного уборщика мусора для рантайма на C}. Для этого необходимо:
\begin{enumerate}[label=\arabic*.]
    \item Разобраться с концепцией автоматического управления памятью, принципами работы сборщиков мусора и особенностями рантаймов языков программирования.
    \item Изучить современные алгоритмы сборки мусора, такие как \textit{Reference Counting}, \textit{Mark-Sweep}, \textit{Mark-Compact}, \textit{Copying GC} и \textit{Generational GC}.
    \item Реализовать собственный сборщик мусора для рантайма на языке C, выбрав оптимальный алгоритм и протестировав его эффективность.
\end{enumerate}

\subsection{Актуальность и значимость}

Несмотря на существование развитых систем управления памятью, понимание принципов работы сборщиков мусора имеет важное значение для оптимизации программного кода, повышения производительности и написания безопасных программ. Разработка собственного GC позволяет глубже изучить принципы работы автоматического управления памятью и исследовать возможные улучшения существующих подходов.

В частности, данный проект будет полезен для:
\begin{itemize}
    \item \textbf{Изучения механизмов управления памятью} в современных языках программирования.
    \item \textbf{Проектирования эффективных алгоритмов очистки памяти} с учетом производительности и оптимизации.
    \item \textbf{Разработки инструментов для языков с автоматическим управлением памятью}, что может быть полезно при создании новых языков программирования или расширении существующих.
\end{itemize}

\subsection{Основные результаты работы}

В ходе проекта будут изучены теоретические основы автоматического управления памятью и реализован собственный сборщик мусора для рантайма на C. Разработанный GC будет протестирован, а его эффективность сравнена с классическими алгоритмами.

\subsection{Структура работы}

В первой части работы будет проведён обзор существующих алгоритмов сборки мусора, рассмотрены их преимущества и недостатки. Во второй части будет описан процесс проектирования и реализации собственного GC, включая архитектурные решения и особенности работы с памятью на C. В третьей части будет проведено тестирование и анализ эффективности реализованного решения, а также сравнение с традиционными подходами. В заключении подводятся итоги работы и предлагаются возможные направления дальнейшего развития проекта.

